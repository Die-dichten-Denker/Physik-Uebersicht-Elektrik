\documentclass[a4paper]{article}

\usepackage{times}
\usepackage[ngerman]{babel}
\usepackage[T1]{fontenc}
\usepackage{amsmath}
\usepackage{amsfonts}
\usepackage{graphicx}
\usepackage{authblk}
\usepackage{fancyhdr}

\pagestyle{fancy}
\renewcommand{\headrulewidth}{0,6pt}
\fancyhead[L]{\leftmark}
\fancyhead[R]{\thepage}

\title{\Huge{Elektrik\\Lernzettel}}
\date{\today}
\author{\quad\\Baden, Julian\\Hagemann, Florian\\\quad\\Gymnasium Mellendorf\\ABI Jahr 2027}

\begin{document}

\maketitle
\thispagestyle{empty}
\newpage
\tableofcontents \thispagestyle{empty}
\newpage
\pagenumbering{arabic}

%________________________________________________________________________________________________

\section{Formelsammlung}
\subsection{Einheiten}

\begin{center}
    \begin{tabular}{ p{4cm} p{4cm} p{4cm} }
         Stromstärke            & $I$           & $A$                 \\[0,5cm]
         Ladung                 & $Q$           & $C$                 \\[0,5cm]
         Spannung               & $U$           & $V$                 \\[0,5cm]
         Wiederstand            & $R$           & $\Omega$            \\[0,5cm]
         el. Feldstärke         & $\vec{E}$     & $\dfrac{N}{C}$      \\[0,5cm]
         Kapazität              & $C$           & $F$                 \\[0,5cm]
         Flächenladungsdichte   & $\sigma$      & $\dfrac{C}{m^2}$    \\[1cm]
    \end{tabular}
\end{center}


\subsection{Formeln}

\Large
\begin{center} 
    \begin{tabular}{ c c }
        $I = \dfrac{Q}{t}$ \hspace{2cm} &  $U = R \cdot I$    \\[0,5cm]
    \end{tabular}
\end{center} 
\normalsize


\paragraph{}











\end{document}